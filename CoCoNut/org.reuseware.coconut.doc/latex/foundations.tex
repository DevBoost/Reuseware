This section gives an overview of the user interface of Reuseware. The user
interface is integrated into Eclipse and works together with modelling editors
installed in your Eclipse. The Reuseware user interface provides the following
two basic features: \textbf{Sokan Stores} and the \textbf{Fragment Repository
View}. Sokan Stores are folders in the Eclipse workspace that contain the
artifacts Reuseware should work with (i.e., models or code that should be
composes). You can mark any folder in any kind of project as a fragment store, by
selecting the folder and pressing the Activate Fragment Store button in the
toolbar shown in Figure 13.
 
Figure 13 � Fragment Store activation button in Eclipse toolbar Fragments that
are registered in a store are available for reuse in composition programs. Each
fragment has a Unique Fragment Identifier (UFI). The UFI is determined by the
position of the fragment in the store. Fragment stores can also contain
composition programs (*.fc files), composition system definitions (*.csys files)
and reuse extensions (*.rex files). How such files are created is described in
Sections 5.3 and 5.4. The Fragment Repository View (cf. Figure 14) lets you
inspect which fragments are available in your system. Open the view in Eclipse
through Window > Show Views > Other... > Reuseware > Fragment Repository. From
the view you can 1) directly open a fragment by double-clicking it 2) select
fragments you want to reuse in a composition program by pressing the + button.
Another way to inspect and search the repository is the fragment browser
component that can be installed in addition and is described in Section 5.5.


\section{Resource Management with Sokan}



Introduce the features available in Eclipse when Sokan is installed.

\section{Language Development with EMF}

Link to external sources for metamodelling and concrete syntax development

- EMF Book

- GMF Guide

- EUGENIA Homepage

- EMFText Guide